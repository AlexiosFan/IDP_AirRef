\documentclass[12pt,letterpaper]{article}
\usepackage{amsmath}
\usepackage{amssymb}
\usepackage{amsthm}

\usepackage{algorithm}
\usepackage{algpseudocode}
\usepackage{xcolor}

\newcommand*{\bigo}{\mathcal{O}}
\newcommand*{\R}{\mathbb{R}}
\newtheorem{theorem}{theorem}[section]
\newtheorem{definition}[theorem]{Definition}
\newtheorem{lemma}[theorem]{Lemma}
\newtheorem{corollary}[theorem]{Corollary}
\newtheorem{observation}[theorem]{Observation}



\title{A Convex Hull for Airplane Refueling Problem(subject to change)}
\author{Zixuan Fan}

\begin{document}

\maketitle

\section{Motivation}
This project aims to a description of the convex hull for the linear programming description of the 
airplane refueling problem. The problem lies in the field of mathematical optimization, 
which is an interdisciplinary branch of mathematics and computer science. In addition, 
an aspect of theorectial computer science is also involved: the computational complexity of the problem 
is a foundamental goal for the problem, and is worthing studying in the project. The project suits me well for two reasons.
\begin{itemize}
    \item My application area in bachelor's study was mathematics, where I took the course Einführung in die Optimierung. This course provides proper background knowledges for the project. 
    \item My main interest in master's study is theorectical computer science. The project is a good opportunity for me to combine my interest in mathematics and computer science.
\end{itemize}

\section{Project Description}
\subsection{Theoretical Foundations}
A natural language description of air plane refueling problem by \cite{puzzle}
\begin{quotation}
    Given a sequence of airplanes filled with fuel. Suppose they can refill each other while flying.
    What is the longest distance/flying time the last plane can reach?
\end{quotation}
A formal definition by \cite{woeginger2010scheduling} is based on the following assumptions
\begin{itemize}
    \item The airplanes are flying in a straight line.
    \item All airplanes consume the fuel of only one airplane simultaneously
    \item Whenver an airplane runs out of fuel, it lands immediately.
    \item We use flying time instead of distance for the objective function
\end{itemize}
The problem is formulated as follows
\begin{definition}
    Given $n$ planes, each with its capacity $w_j$ and fuel consumption rate $p_j$. What is the maximal flying time the last plane can reach?
\end{definition}
A reduction to scheduling problem was found. Let $C_j$ denote the completion time of job $j$, i.e. the flying time of the plane $i$.
The optimization goal is 
\begin{align*}
    \min -\sum_{j=0}^n w_j / C_j
\end{align*}
Many researchers focus on the precedence of the problem for an efficient algorithm \cite{li2019fast} \cite{vasquez2015airplane},
In this project, I will examine the problem from another aspect: convex hull of the schedule vectors, $(1/C_1, ..., 1/C_n)$. \\
\color{red} Question: Why are there $n!$ such vectors? Is it because there are $n!$ permutations, and each permutation 
corresponds to a unique vector? \color{black}

\subsection{Practical Goals}
Aside from the theoretical aspect, the convex hull needs to be actually computed given an instance of the problem. 
This requires to take advantage of the existing software, Polymake. Polymake is a software for research in polyhedral geometry.
It can help to compute the polytopes and polyhedra, which potentially makes up the desired convex hull in this project. 
The practical goal will be implementing the algorithm for the computation of the convex hull. \\
\color{red} Question: using software itself and C++ programming for an extension?
\color{black}

\section{Course to visit from another subject}
The course I would like to visit is Combinatorial Optimization(MA4502) offered by the department of Mathematics at School of CIT.
This course deals with application of linear programming in many combinatorial problems such as netflows, matching etc. 
In addition, it also studies approximation of NP-hard problems via linear programming. Through learning of the course, 
I believe I will obtain more insights into handling of combinatorial problems, and ideally be able to work on the project. 

\section{Timetable and Milestones: TODO and to discuss}

\bibliographystyle{plain}
\bibliography{citations}
\end{document}