\documentclass[12pt,letterpaper]{article}
\usepackage{amsmath}
\usepackage{amssymb}
\usepackage{amsthm}

\usepackage{cleveref}

\usepackage{algorithm}
\usepackage{algpseudocode}
\usepackage{xcolor}
\usepackage{tikz}
\usepackage{float}

\newcommand*{\bigo}{\mathcal{O}}
\newcommand*{\R}{\mathbb{R}}
\newtheorem{theorem}{theorem}[section]
\newtheorem{definition}[theorem]{Definition}
\newtheorem{lemma}[theorem]{Lemma}
\newtheorem{corollary}[theorem]{Corollary}
\newtheorem{observation}[theorem]{Observation}



\setlength\parindent{0pt}

\title{Computing the convex hull of solution vectors to the airplane refueling problem to find a general description}
\author{Zixuan Fan}

\begin{document}

\section{Motivation}

\section{Mathematical Backgrounds}
The Airplane Refueling Problem is combinatorial problem that is 
neither known to be NP-complete nor polynomially solvable. 
\begin{definition}
    Given $n$ planes each with a tank $w_j$ and consumption rate $p_j$,
    what is the longest time/distance that this fleet of planes can fly,
    if they can refuel each other?
\end{definition}
An intuitive idea is consuming the tank of only one plane 
at a time, and drop it out after its tank is empty. This approach 
finds a \textit{drop out ordering} $\sigma$
that maximizes the objective function 
\begin{align*}
    \max \sum_{j=1}^{n} \dfrac{w_{\sigma(j)}}{\sum_{k = j}^n p_{\sigma(k)}}
\end{align*}
The problem can also be formalized as a scheduling problem, proposed by Woeginger. 
In context of scheduling, $\sigma$ can be viewed as permutation, the reverse of ordering, 
while we can also change the objective function to a simpler form. 
\begin{align*}
    \sum_{j = 1}^n \dfrac{w_{\pi(j)}}{\sum_{k = 1}^j p_{\pi(k)}} =  \sum_{j = 1}^n \dfrac{w_{j}}{C_j}
\end{align*}
On the basis of this, we are also able to introduce a mathematical programming formulation.
\begin{align}
    &\max \sum_{j = 1}^n \dfrac{w_j}{C_j} \label{eq:mathprog} \tag{A} \\ 
    \text{s. t.\ }&  C_{\pi(j+1)} - C_{\pi(j)} = p_{\pi(j)} \forall j \nonumber
\end{align}
As we know, a linear program or a convex semi-definite program is solvable using various interior-point methods. 
Since the mathematical program is neither linear nor convex semi-definite, our goal is to find a reduction the 
realizes this goal. On the otherhand, we are also interested in formulate the problem as an integer program, 
for it also provides us with some other possible solution approaches. 

\section{First Attempt in Theory}
Our first attempt focused on the fact that the solution space of the formulation \ref{eq:mathprog}
is a polytope based on permutations. Each permutation generates a tuple of $(1/C_1, 1/C_2, ..., 1/C_n)$.
Since $w_j$'s are given constants, a straightforward solution is simply traversing all $n!$ permutations. 
\subsection{Permutahedron}
A permutahedron is a polytope that is defined by permutations. 
In the permutahedron of the $k$-th order $P_k$, each vertex is a permutation of $k$ elements.
Hence there are $k!$ many vertices in $P_k$. For example, a hexagon is a second-order permutahedron. \\
Although the number of faces is not polynomial in $n$,
some symmetric property was discovered, and can be used for optimization. Some researchers(cite later)
have found an extended formulation $\R^{k^2+2k} \rightarrow \R^{k}$, which allows us 
to solve the LP on $P_k$ in polynomial time. 

\subsection{Details in attempts}
Elaborate formulation later: the symmetric property does not exists in 
our polytope. Using the similar formulation, we end up with a 
semi-definite program that has a convex feasible region, 
but concave objective function. We did not go further on this direction. 
However, two possiblility remains to be explored:
\begin{itemize}
    \item Can we find an extended formulation by changing the formulation for permutahedron slight?
    \item Can we try some classical methods like interior-point method on this formulation, what is the result?
\end{itemize}

\section{Second Attempt by Computation}
After this attemtp in theory, we decided to move to do some computation work. 
For simplicity, python was chosen to generate some randomized inputs and scripts for 
the computation tool, polymake. 

\subsection{Input Generation}
As we have analyzed in the first attempt, we want to compute the polytope of the all solution vectors 
\begin{align*}
    s = (\dfrac{1}{C_1}, \dfrac{1}{C_2}, ..., \dfrac{1}{C_n})
\end{align*}
Since the completion time is based on the permutation $\pi$, we need a few steps to generate all possible 
vectors. 
\begin{enumerate}
    \item Generate a randomized/special consumption time $p_j$
    \item Pick a permutation $\pi$, arrange the consumption time in the order of $\pi$
    \item Compute the completion time by $C_1 = p_1$ and $C_{i+1} = C_i + p_{i+1}$
    \item Compute the inverse of $C_j$
    \item Repeat until all permutations are visited
\end{enumerate}
This results in $n!$ vertices in $\R^n$. A straightforward traveral on the vertices is absolutely not 
polynomial-time, but it may help us to find some polynomial-time structure within the polytope.

\subsection{Usage of Polymake}

\subsection{Results}
Although the computation of the polytope is polynomial-time in the number of vertices, our computation takes 
a significant amount of time. The reason is that the number of vertices is factorial in $n$. 
The computation of a 7-dimensional polytope does not terminate within 24 hours. Thus, we establish a large amount of 
computation only in $n = 4$ and $n = 5$. Some computation of $n = 6$ was also done, but a large amount of computation 
is not applicable. 
\begin{table}
    \centering
    \begin{tabular}{||c | c | c ||}
        \hline 
        $n$ & Number of Facets & Number of Unique Values\\
        \hline 
        \hline 
        2 & 2 & 1\\
        \hline 
        3 & 8 & 1\\
        \hline 
        4 & 67, 68, 69, 70 & 4\\
        \hline 
        5 & $700 \sim 800$ & $\geq 66$\\
        \hline 
        6 & $10000 \sim 12000$  & Unknown\\
        \hline 
    \end{tabular}
    \caption{Number of facets in the polytope for $n \leq 6$}
\end{table}
Starting from $n = 4$, the number of facets is no longer a unique number. 
Hence we guess that the number of facets is somehow related to the input chosen. 
Unfornately, a direct pattern cannot be observer for $n=5$. 
Thus, we try to use machine learning to help us find some pattern. 


\section{Third Attempt by Machine Learning}
In this section, we would propose some machine learning techniques to see if we can find some property that 
either justifies the our polynomial-time prediction or gives some analytical description of the exponential-time behavior. \\
Fortunately, the problem is already geometric, simple unsupervised learning techniques like clustering
can tell us some direct information. In addition, the behavior of the polytope is also related to the input chosen 
as consumption rate. Supervised learning techniques like decision tree, may help us to find a pattern
that either helps us decide the behavior or a subclass of problems that is solvable in polynomial-time.

\subsection{Regression}

\subsection{Clustering}

\subsection{Decision Trees}

\section{Progress during the vacation}
To be discussed in the meeting by the start of November
\subsection{Machine Learning}
Tree models worked pretty well. The visualization results seem to be interesting. 
Would be interesting to employ some quadratic classifier. Default quadratic discrimamt analysis do not work well \\
Discovery in data set: $x_i - x_j = 0$ separates all vertices by a half for all $i, j$.



\subsection{Formulation with Permutahedron}
\begin{align*}
    \max w^Tx& \text{ s.t. } \\
    x_i y_i &= 1 \ \forall i \\ 
    \sum_{i \in [n]} z_{ij} &= 1 \forall j \\
    \sum_{j \in [n]} z_{ij} &= 1 \forall i \\
    \Pi &= (z_{ij}), S = \begin{pmatrix}
        1 & \cdots & 0 \\ 
        \cdots & \cdots & \cdots \\ 
        1 & \cdots & 1
    \end{pmatrix} \\
    y &= \Pi^T S \Pi p
\end{align*}
This is a convex program, if $0 \leq z_{ij} \leq 1$. 
Apparently, computing this program in polynomial time gives 
us an approximation in some form. Is it worth doing so?

\subsection{Formulation with Vasquez's Idea}
Local and Global precedence can be computed upon getting input in polynomial time. 
For the simple case, where the global precedence can be decided directly, we don't need 
to take a second look. However, for the last case, where the global precedence is decided by the 
$t^*$ a binary selection is need. Based on this idea, we can also formulate a convex program.  \\
In this formulation, we assume all global precedence cannot be decided. $z_{ij}$ stands for the precedence. 
If $z_{ij} = 1$ then $i \prec_g j$ and vice versa.
\begin{align*}
    &\max \sum w^T x \text{ s.t. } \\
    &x_iy_i = 1 \\
    & (y_i - t^*_{ij} - p_i) z_{ij} \leq 0, z_{ij} \in \{-1, 1\} \\
    &2y_i = \{2p_i\} + \sum_{j \not= i} p_j (1+z_{ij})
\end{align*}
Again this is a discrete problem. Notice that $z_{ij}$ can be transformed into the sign function of $t^*_{ij} - y_i$.
Here is what we do 
\begin{align*}
    &\max \sum w^T x \text{ s.t. } \\
    &x_iy_i = 1 \\
    & z_{ij} = y_i - t^*_{ij} - p_i \forall i < j \\
    & z_{ij} = -z_{ji} \forall i > j \\
    &2y_i = \{2p_i\} + \sum_{j \not= i} p_j (1+\frac{z_{ij}}{|z_{ij}|})
\end{align*}
We are able to simplify the last constraint even a bit more by removing $y$'s and $z$'s
\begin{align*}
    &\max \sum w^T x \text{ s.t. } \\
    &(\frac{1}{x_i} - t^*_{ij} - p_i)(\frac{1}{x_j} - t^*_{ij} - p_j) \leq 0 \forall i < j \\
    &\dfrac{2}{x_i} = \{2p_i\} + \sum_{j \not= i} p_j \big( 1 + \dfrac{\frac{1}{x_j} - t^*_{ij} - p_j}{|\frac{1}{x_j} - t^*_{ij} - p_j|} \big)
\end{align*}
It is easy to see that the second constraints are polynomial functions w.r.t. $x_i$ only. 
And there $n$ solutions for each of $x$. Mor}{|\frac{1}{x_j} - t^*_{ij} - p_j|} \big)
\end{equation*}
Suppose we solve all of these equations to obtain all $n$ solutions for each $x_i$.
If arranged in decreasing order, they will correspond to their position in the permutation. 
This where we can employ the formulation of permutahedra again. \\
For the $j$-th solution of $x_j$ in decreasing order, we donote it as $x_{ij}$
\begin{align*}
    &\max \sum w^T x \text{ s.t. } \\
    & 0 \leq z_{ij} \leq 1 \forall i, j \\
    &\sum_{i \in [n]} z_{ij} \leq 1 \forall j \\
    &\sum_{j \in [n]} z_{ij} \leq 1 \forall i \\
    &x_i = \sum_{j \in [n]} x_{ij} z_{ij} 
\end{align*} 
This is a linear program, and can be solved in polynomial time. Recall that $x_{ij}$
are $n^2$ constant values obtained in polynomial time. If this formulation is indeed correct, 
we then over. Thus, we now need a computational proof for it. 

\end{document}